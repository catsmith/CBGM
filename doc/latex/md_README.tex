This is my implementation of the C\+B\+GM. It was designed for testing and changing the various algorithms and is not (therefore) the fastest user-\/facing package. The idea was that everything be calculated from scratch each time -\/ although in later development I added a Genealogical Coherence Cache for convenience (and speed). If you change the input data then remember to delete the old cache...

\subsection*{I\+N\+S\+T\+A\+LL }

1) You need these packages installed (these are the Ubuntu names)\+:
\begin{DoxyItemize}
\item openmpi-\/bin
\item libopenmpi-\/dev
\item graphviz
\item python3-\/dev
\item libgraphviz-\/dev 2) You need a python 3.\+5 virtualenv, using the provided requirements.\+txt
\end{DoxyItemize}

\subsection*{U\+SE }

The main program is ./bin/cbgm. Run {\ttfamily ./bin/cbgm -\/-\/help} for info, and for info about the subcommands do, for example, {\ttfamily ./bin/cbgm combanc -\/-\/help}.

Similarly, see the help for the other programs in bin.

\subsection*{D\+O\+C\+U\+M\+E\+N\+T\+A\+T\+I\+ON }

To create the code documentation, run {\ttfamily doxygen doxygen.\+conf}

\subsection*{T\+E\+S\+T\+I\+NG }

To run the unit tests, run {\ttfamily ./test.sh}

Also see the \char`\"{}examples\char`\"{} folder, including the R\+E\+A\+D\+M\+E.\+sh in there, which can be executed with bash to execute the common C\+B\+GM features. 